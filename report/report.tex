\documentclass[10pt,a4paper]{article}
\usepackage[utf8]{inputenc}
\usepackage{amsmath}
\usepackage{amsfonts}
\usepackage{amssymb}
\author{Myles Lee}
\title{Repository Knowledge Transfer in Automatic Optimisation of
Reconfigurable Designs}
\begin{document}
\maketitle
\section*{Introduction}
\subsection*{Problem}
The objective is to optimise a design $T$, by finding the optimial design $x_T^*$ using the performace metric $f_T$, using the least number of evaluations of $f_T$. $T$ has $n_T$ reconfigurable parameters, the design space $\mathbb{X}_T$ is defined to be the mathematical Cartesian product of $n_T$ real intervals. $f_T$ is expensive to evaluate and cannot be assumed that is differentable or convex. Therefore, computing $x_T^*$ is expensive. Finding the true optimial design maybe too costly, and a near-optimial design may be used.
\subsection*{Previous Work}
To minimise this cost, a previous effort \cite{ego} introduced another design $S$. Using observations $\{(x_1,y_1),...\}$ from $f_S$ and $x_S^*$, $x_T^*$ can be approximated.
\section*{Contributions}
The main contributions are summarised below:
\begin{enumerate}
\item Improvement of knowledge transfer to predict obserations of $f_T$, using a \emph{Transferrable Gaussian Process}
\item Using multiple designs to improve knowledge transfer
\item Extension of Will's code base with implementation of the above concepts
\end{enumerate}
\subsection*{Transferable Gaussian Process}
Previously, we use observations $O_T=\{(x_1,y_1),...\}$ from $f_T$ to train a surrogate $\hat{f_T}$ so that $\hat{f_T}(x)\approx f_T(x)$. Let us define use the syntax $p_{O_T}$ to represent $\hat{f_T}$. From \cite{ego}, a linear mapping from $y_S$ to $y_T$ is used.

This not practical in some cases.

\section*{Evaluation}
a load of graphs here
\section*{Future Work}

\begin{thebibliography}{9}
\bibitem{ego}
Kurek, M and Deisenroth, MP and Luk, W and Todman, T,
2016,
\textit{Transfer in Automatic Optimisation of Reconfigurable Designs},
24th IEEE International Symposium on Field-Programmable Custom Computing Machines (FCCM), Publisher: IEEE, Pages: 84-87

\end{thebibliography}
\end{document}